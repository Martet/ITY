\documentclass[11pt]{article}
\usepackage[utf8]{inputenc}
\usepackage[czech]{babel}
\usepackage[IL2]{fontenc}
\usepackage{xurl}
\usepackage{amssymb,amsmath}
\usepackage[a4paper, total={17cm, 24cm}, left=2cm, top=3cm]{geometry}

\begin{document}

\begin{titlepage}
\begin{center}
{\Huge \textsc{Vysoké učení technické v Brně}\\
\bigskip
{\huge \textsc{Fakulta informačních technologií}}} \\
\vspace{\stretch{0.382}}
{\LARGE Typografie a publikování\,--\,4. projekt}\\
\medskip
{\Huge Citace} \\
\vspace{\stretch{0.618}}
\end{center}
{\Large \today \hfill Martin Zmitko (xzmitk01)}
\end{titlepage}

\section{Typografie}
Typografie se zabývá písmem a~způsoby, jak ho vysázet na stránku.
Hlavním cílem typografie je, aby si jí čtenář vůbec nevšiml \cite{May:2018}.
To totiž znamená, že je pro něj text snadno čitelný a~logicky strukturovaný.
Toho lze dosáhnout dodržováním typografických zásad.
K~těm patří konzistence, střídmost v~použití efektů jako zvýraznění nebo barev a~správné rozdělení volného místa \cite{Archer:2011}.
Typografií se zabýva již dlouho vydávaný český časopis Typografia \cite{Typografia}.

\section{Nástroje pro sázení dokumentů}
K~sázení dokumentů dle správných typografických zásad můžeme využít mnoho nástrojů.
Asi každému je známý program Microsoft Word, na sázení profesionálně vypadajících dokumentů se ale nehodí, viz~\cite{Freeman:2017}.
Mezi nástroje přímo dělané pro sázení profesionálních rozsáhlejších dokumentů patří např. Adobe InDesign, ten ale stojí několik set dolarů\cite{Reedsy:2018} a~není tedy správnou volbou pro každého.

Jako nejlepší nástroj pro sázení dokumentů nelze doporučit nic jiného než \LaTeX.
Je to open-source systém pro sázení dokumentů, přičemž se dokumenty píšou v markupovém jazyce, možná hůře srozumitelném neznalému.
Pro podrobnější seznámení s~\LaTeX em se vyplatí přečíst si knihu jako např.~\cite{Rybicka:2003} nebo anglicky~\cite{Lamport:1994}. 

\section{\LaTeX}
\subsection{Citace}
Citace jsou při psaní dokumentů velmi důležité, protože vzájemně propojují informace z~různých zdrojů a~jasně odlišují vlastní a~převzatý text.
V~České republice platí citační norma ČSN~ISO~690, podle které by se mělo citovat.
V~\LaTeX u se na citace používá nástroj \textsc{Bib}\TeX , například s~českým stylem~\cite{Pysny:2009}.

\subsection{Matematika}
Jednou z~hlavních předností \LaTeX u je možnost sázení matematických formulí a~symbolů s~jednoduchostí a~přesností nenalezitelnou u~žádného jiného podobného nástroje \cite{Wyrwolova:2014}.
Ukázka jednoduchosti sázení matematického výrazu:
\bigskip

\noindent\verb|\[\Re{z} =\frac{n\pi \dfrac{\theta +\psi}{2}}{|\\
\verb|\left(\dfrac{\theta +\psi}{2}\right)^2 + \left( \dfrac{1}{2}|\\
\verb|\log \left\lvert\dfrac{B}{A}\right\rvert\right)^2}\]|
\bigskip

\noindent Tento výraz převzatý z~\cite{Downes:2002} se vysází následovně:
\[\Re{z} =\frac{n\pi \dfrac{\theta +\psi}{2}}{
\left(\dfrac{\theta +\psi}{2}\right)^2 + \left( \dfrac{1}{2}
\log \left\lvert\dfrac{B}{A}\right\rvert\right)^2}\]

\newpage
\bibliographystyle{czechiso}
\bibliography{proj4}

\end{document}
